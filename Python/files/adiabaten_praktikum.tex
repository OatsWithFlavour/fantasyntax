\documentclass{protokoll}

\praktikumstitel{Grundpraktikum Physik}
\versuchstitel{M10 „Federpendel“}
\autoren{Hannes Honert, Alexander Kühn}
\studiengang{TKS, EIT (DIPLOM)}
\datum{8. Mai 2025}

\begin{document}
\section{Aufgabenstellung}
1. Der Adiabatenexponent von Luft ist mit Hilfe
eines Expansionsexperimentes nach 
Cl\'{e}ment-Desormes zu bestimmen.
\\
2. Der Adiabatenexponent von Luf ist aus einem
Schwingungsexperiment nach Rüchardt zu 
bestimmen.

\section{Theoretische Grundlagen}
Der Zustand eines idealen Gases wird eindeutig
durch das Volumen $V$ den Druck $p$ und die 
Temperatur $T$ beschrieben:
\gl{pV = nRT}
mit der $n$ der Mohlzahl und 
$R = 8.314 \ bruch{J}{mol K}$ der
universellen Gaskonstante.

Die Änderung der inneren Energie 
$\bruch{dU}{dt}$ eines
Systems wird durch Summe der Änderungen aus
Wärme $\bruch{dQ}{dt}$ und
am System verrichtete Arbeit $\bruch{dW}{dt}$
erzeugt:
\gl{\bruch{dU}{dt} = \bruch{dQ}{dt} + \bruch{dW}{dt},
dU = dQ + dW}

Bei adiabatischen Prozessen wird die am 
System verrichtete Arbeite viel schneller 
geändert als die Wärme sich ändert:
\gl{dQ << dW}
Deshalb die Annahme:
\gl{dU = dW}

Wird das Volumen bei gleichen Druck 
verringert wird Arbeit am System verrichtet:
\gl{dW = -pdV}



Die innere Energie eines idealen Gases
$U$ wird durch die spezifische
Wärekapazität bei konstanten Druck $C_V$ die Mohlzahl 
$n$ und durch die Temperatur $T$ bestimmt:
\gl{U = nC_VT}
für kleine Temperaturänderungen $dT$:
\gl{dU = nC_VdT}

Wird die innere Energie eines adiabatischen Prozesses mit der inneren Energie
eines idealen Gases verglichen erhält man:
\gl{dU = dW = -pdV = nC_VdT}
Wird der Druck $p$ durch die Zustandsgleichung idealer Gase ersetzt:
\gl{-\bruch{nRT}{V}dV = nC_VdT,
-\bruch{R}{C_V}\bruch{1}{V}dV= \bruch{1}{T}dT,
-\bruch{R}{C_V}ln(V) + C = ln(T),
C = \bruch{R}{C_V}ln(V) + ln(T),
e^C = V^\bruch{R}{C_V}T = const.
}

Aus den Grundgleichungen idealer Gase kann folgender abgeleitet werden:
\gl{C_P = C_V + R, \kappa = \bruch{C_P}{C_V} = 1 + \bruch{R}{C_V}}

So erhält man eine der Adiabatengleichungen mit dem Adiabatenexponenten $\kappa$:

\gl{TV^{\kappa - 1} = const.}

Analog können auch die anderen Adiabatengleichungen hergeleitet werden:

\gl{pV^\kappa = const., T^\kappa p^{1 - \kappa} = const.}

Der Adiabatenexponent $\kappa$ ist das Verhältnis der spezifischen
Wärmekapazität bei konstanten Druch $C_P$ und der spezifischen  Wärmekapazität
bei konstantem Volumen $C_V$:

\gl{\kappa = \bruch{C_P}{C_V} = \bruch{f + 2}{f}}

Mit $f$ ist die Anzahl der Freiheitsgrade des Moleküls. Einatomige Gase besitzen
nur drei Freiheitsgrade, zweiatomige Gase besitzen fünf Freiheitsgrade.

\section{Versuchsbeschreibung}

In diesem Versuch wurde der Adiabatenexponent der Luft auf zwei unterschiedliche 
Arten bestimmt:

\begin{enumerate}
    \item Mittels eines \textbf{Expansionsversuchs nach Clément-Desormes}, 
    bei dem die Druckänderung nach adiabatischer Expansion und anschließender 
    isochorer Erwärmung in einer luftgefüllten Glasflasche gemessen wurde.
    \grafik{Aufbau Expansionsversuch}
    \item Mithilfe eines \textbf{Schwingungsversuchs nach Rüchardt}, bei 
    dem ein luftdicht beweglicher Kolben auf einem Luftpolster zu harmonischen 
    Schwingungen angeregt wurde.
    \grafik{Aufbau Schwingungsversuch}
\end{enumerate}

Ziel war es, aus den jeweiligen physikalischen Beziehungen 
(lineare Regression Druckverhältnisse oder aus $T^2, V$) den Adiabatenexponenten 
$\kappa$ der Luft zu berechnen.

\section{Versuchsdurchführung}

\subsection{Expansionsversuch}

Zunächst wurde im Vorversuch die notwendige Wartezeit zur Erreichung des 
Temperaturgleichgewichts ermittelt. Dazu wurde mithilfe einer Handpumpe ein
Überdruck von ca. $98{,}4 \, \mathrm{mbar}$ in der Glasflasche erzeugt, der 
Absperrhahn geschlossen und der Druck im $15 \, \mathrm{s}$-Takt über $5 \, 
\mathrm{min}$ aufgezeichnet. Danach wurde das Tellerventil kurz geöffnet und 
erneut $5 \, \mathrm{min}$ lang im selben Takt der Druck gemessen. Die 
aufgenommenen Daten wurden mittels nichtlinearer Regression (Exponentialfunktion 
mit Asymptote) im Programm \textit{PhysPract} ausgewertet. Daraus ergaben sich 
die minimalen Wartezeiten:

\begin{itemize}
    \item $150 \, \mathrm{s}$ nach dem Aufpumpen
    \item $100 \, \mathrm{s}$ nach dem Druckausgleich
\end{itemize}

Im eigentlichen Versuch wurde der Startüberdruck variiert
10-100 mbar in $\approx 10\, \mathrm{mbar}$-Schritten.
Für jeden Wert wurden der Gleichgewichtsdruck $p_1$ und der Enddruck $p_2$ nach 
den oben bestimmten Wartezeiten gemessen.

\subsection{Schwingungsversuch}

Für den Rüchardt’schen Versuch wurde ein Kolben in einem Glaskolbenprober auf 
definierte Höhen gebracht, sodass ein messbares Luftvolumen eingeschlossen war. 
Durch Herausziehen und Loslassen wurde der Kolben zur Schwingung gebracht. Die 
Schwingungsdauer für $n = 10$ Perioden wurde für die Volumina $20, 40, 60, 70, 
90 \, \mathrm{ml}$ bestimmt.  
Die Masse des Kolbens betrug $m = (125{,}94 \pm 0{,}02) \cdot 10^{-3} \, 
\mathrm{kg}$, die Raumtemperatur lag bei $23{,}5\, ^\circ\mathrm{C}$ und der 
Luftdruck bei $p_0 = (729{,}5 \pm 0{,}25)\, \mathrm{Torr} \approx (97258{,}67 
\pm 33{,}35)\, \mathrm{Pa}$. Der Rohrdurchmesser war $d = (31{,}10 \pm 0{,}05) 
\cdot 10^{-3}\, \mathrm{m}$.




\section{Messwerte}

\begin{spalten}

\zeilentabelle{Vorversuch_1_Expansion.csv}
\anderespalte
\zeilentabelle{Vorversuch_2_Expansion.csv}

\end{spalten}

\zeilentabelle{Expansionsversuch.csv}
\zeilentabelle{Schwingungsversuch.csv}




\section{Auswertung und Ergebnis}

\subsection{Expansionsversuch nach Clément-Desormes}
\grafik[1]{Vorversuch 1 zum Expansionsversuch}
\grafik[1]{Vorversuch 2 zum Expansionsversuch}
\grafik[1]{Expansionsversuch}


Die gemessenen Druckwerte $p_1$ und $p_2$ wurden im Programm \textit{PhysPract} 
dargestellt und durch eine Ausgleichsgerade angepasst. Aus dem 
Anstieg $S = 0{,}260$ ergibt sich mit der Formel

$$
\kappa = \frac{1}{1 - S}
$$

der Wert:

$$
\kappa = \frac{1}{1 - 0{,}260} = 1{,}351
$$



\subsection{Schwingungsversuch nach Rüchardt}
\grafik[1]{Schwingungsversuch}

Für fünf unterschiedliche Volumina wurde jeweils die Schwingungsdauer 
$T$ für $n = 10$ Perioden gemessen. Aus den $T^2$-Werten wurde eine lineare 
Regression $T^2 = f(V)$ durchgeführt. Der Regressionsanstieg beträgt:

$$
S = (64{,}19 \pm 1{,}74) \, \mathrm{s^2/m^3}
$$

Mit
\begin{align*}
m &= (125{,}94 \pm 0{,}02) \cdot 10^{-3} \, \mathrm{kg} \\
d &= (31{,}10 \pm 0{,}05) \cdot 10^{-3} \, \mathrm{m} \\
p_0 &= (97258{,}67 \pm 33{,}35) \, \mathrm{Pa}
\end{align*}

und der Formel

$$
\kappa = \frac{64 \cdot m}{S \cdot p_0 \cdot d^4}
$$

ergibt sich:

$$
\kappa = 1{,}380
$$

Die kombinierte relative Unsicherheit ergibt sich aus:

$$
\frac{\Delta \kappa}{\kappa} = \left| \frac{\Delta m}{m} \right| + 
\left| \frac{\Delta S}{S} \right| + \left| \frac{\Delta p_0}{p_0} \right| + 
4 \cdot \left| \frac{\Delta d}{d} \right| \approx 0{,}0340
$$

Dabei gelten:

\begin{align*}
\frac{\Delta m}{m} &= \frac{0{,}00002 \, \mathrm{kg}}{0{,}12594 \, \mathrm{kg}} \approx 0{,}000159 \\
\frac{\Delta S}{S} &= \frac{1{,}74 \, \mathrm{s^2/m^3}}{64{,}19 \, \mathrm{s^2/m^3}} \approx 0{,}0271 \\
\frac{\Delta p_0}{p_0} &= \frac{33{,}35 \, \mathrm{Pa}}{97258{,}67 \, \mathrm{Pa}} \approx 0{,}000343 \\
4 \cdot \frac{\Delta d}{d} &= 4 \cdot \frac{0{,}00005 \, \mathrm{m}}{0{,}03110 \, \mathrm{m}} \approx 0{,}00643
\end{align*}

Daraus ergibt sich:

$$
\frac{\Delta \kappa}{\kappa} \approx 0{,}000159 + 0{,}0271 + 0{,}000343 + 0{,}00643 = 0{,}0340
$$

Die absolute Unsicherheit von $\kappa$ beträgt:

$$
\Delta \kappa = \kappa \cdot \frac{\Delta \kappa}{\kappa} = 1{,}380 \cdot 0{,}0340 \approx 0{,}047
$$

Somit ergibt sich als Endergebnis:

$$
\boxed{\kappa = (1{,}380 \pm 0{,}047)}
$$




\section{Fehlerdiskussion}

Im Expansionsversuch wurde der Adiabatenexponent zu $\kappa = 1{,}351$ bestimmt. 
Die Differenz zum theoretischen Wert von $\kappa_{\text{theo}} = 1{,}40$ 
(für trockene Luft als zweiatomiges Gas) ist durch 
\textbf{systematische Effekte} erklärbar. Besonders hervorzuheben 
ist der Einfluss von \textbf{Wasserdampf}, einem dreiatomigen Molekül mit 
sechs Freiheitsgraden, was den Wert in Richtung $\kappa \approx 1{,}33$ senkt. 
Der gemessene Wert liegt somit plausibel im Übergangsbereich zwischen trockener 
und feuchter Luft.

Im Schwingungsversuch konnte mit höherer Genauigkeit ein Wert von

$$
\kappa = (1{,}380 \pm 0{,}047)
$$

bestimmt werden, der den Literaturwert von $\kappa = 1{,}40$ innerhalb der 
Unsicherheit einschließt. Die \textbf{größten Unsicherheiten} ergaben sich 
durch die Regressionsberechnung der Ausgleichsgeraden sowie durch die vierte 
Potenz des Rohrdurchmessers in der Formel. Die Schwingungsdauer wurde jedoch 
mit hoher Präzision gemessen, was den relativ kleinen Fehler erklärt.

Zusätzlich zu den statistisch erfassten Unsicherheiten könnten 
\textbf{systematische Fehler} wie Reibung im Rohr, kleine Undichtigkeiten 
oder Ablesefehler beim Volumen auftreten, die sich jedoch nur geringfügig auf 
das Ergebnis auswirken sollten.




\section{Anhang}



\end{document}
%