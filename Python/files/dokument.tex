\documentclass{protokoll}
\praktikumstitel{Grundpraktikum Physik}
\versuchstitel{M2 Schwerebeschleunigung}
\autoren{Hannes Honert, Alexander Kühn}
\studiengang{TKS, EIT (DIPLOM)}
\datum{27. Mai 2025}
\begin{document}
\section{Aufgabenstellung}


\begin{enumerate}
\item Die Schwerebeschleunigung der Erde ist aus dem freien Fall 
einer Stahlkugel zu bestimmen. Der Einfluss der Luftreibung 
ist abzuschätzen
\item Die Schwerebeschleunigung ist weiterhin mithilfe einer Atwoodschen 
Fallmaschine zu bestimmen. Effektive Masse und Lagerreibung der 
Umlenkrolle sind anzugeben.
\end{enumerate}


\section{Theoretische Grundlagen}

Die Erdbeschleunigung $g$ liegt nach Tafelwerk bei $g = 9.80665 \frac{m}{s^2}$.

\subsection{Erdbeschleunigung ohne Luftreibung:}


Die Erdbeschleunigung $g$ wird bei Vernachlässigung der Luftreibung wie folgt definiert:

\begin{equation}
 g = \frac{2s}{t}^2 
\end{equation}


wobei $s$ der Weg und $t$ die Fallzeit ist.


\subsection{Erdbeschleunigung mit Luftreibung:}


Der Weg des freien Falls mit Einfluss der Luftreibung 
wird gemäß der Versuchsanleitung so definiert:

\begin{equation}
 s(t) = \frac{1}{2} g t^2 - \frac{1}{12} c g^2 t^4 
\end{equation}

umgestellt nach $2s(t)$:

\begin{equation}
 2s = g t^2 (1 - \frac{1}{6} c g t^2) 
\end{equation}

nach $g$ durch Nullstellen berechnen:

\begin{equation}
 0 = \frac{1}{2} g t^2 - \frac{1}{12} c g^2 t^4 - s 
\end{equation}
\begin{equation}
 0 = g^2 - \frac{6}{ct^2} g + \frac{12s}{ct^4} 
\end{equation}
\begin{equation}
 \frac{g_1}{2} = \frac{3}{ct}^2 plusminus (wurzel(9 - 12cs))/(ct^2) 
\end{equation}
der positive Wert:

\begin{equation}
 g = (3 + wurzel(9 - 12cs))/(ct^2) 
\end{equation}

Eine Formel für die Erdbeschleunigung $g$ mit dem Einfluss der Luftreibung entsteht.

Dabei wird die Konstante $c$ wird bestimmt aus:

\begin{equation}
c = \frac{3 rho_L}{20 rho_K r}
\end{equation}

wobei $rho_L$ die Luftdichte, $rho_K$ die Dichte der Kugel 
und $r$ der Radius der Kugel ist.

Die Luftdichte $rho_L$ wird wie folgt bestimmt:

Gegeben ist die Dichte der Luft unter Normbedingungen: 
$ rho_L (p_0, T_0) = 1.2928 \frac{kg}{m}^3, p_0 = 101.3 kPa, 
T_0 = 0gradC = 273.15 K$

Die thermische Zustandsgleichung idealer Gase lautet wie folgt:

\begin{equation}
nRT = pV
\end{equation}

mit $n$ der Stoffmenge, $R$ der universellen Gaskonstante, 
$T$ der Temperatur, $p$ den Druck und $V$ den Volumen.

Die Stoffmenge ist definiert als:

\begin{equation}
 n = \frac{m}{M} 
\end{equation}

mit $m$ der Masse und $M$ der molaren Masse.

$M$ und $R$ sind konstant. $M$ ist konstant, da es sich 
um die gleiche Luft handelt. Setzt man die Gleichungen 
in einander ein und stellt nach den Konstanten um erhält man:

\begin{equation}
 \frac{R}{M} = \frac{pV}{mT} 
\end{equation}

wobei $\frac{m}{V} = rho$ die Dichte darstellt:

\begin{equation}
 \frac{M}{R} = \frac{rho T}{p} = konst.
\end{equation}

Setzt man nun für $ rho = rho_L = 1.2928 \frac{kg}{m}^3, 
T = T_0 = 0gradC = 273.15 K $ und für $ p = p_0 = 101.3 kPa $
erhält man:

\begin{equation}
 \frac{M}{R} = \frac{rho_L * T_0}{p_0} = (1.2928 \frac{kg}{m^3} * 273.15 K)/(101.3 kPa)
= 3.49 * 10^-3 \frac{Ks^2}{m^2} 
\end{equation}

Um den aktuellen Luftdruck zu messen, wird nach $rho$ umgestellt:

\begin{equation}
 rho = \frac{M}{R} \frac{p}{T} 
\end{equation}

hier:

\begin{equation}
 rho_(Luft, aktuell) = 3.49 * 10^-3 \frac{Ks^2}{m^2} * 
p_(Luft, aktuell)/T_(Luft, aktuell) 
\end{equation}

Diese Formel kommt im Versuch zum freien Fall zum Einsatz

\subsection{Atwoodsche Fallmaschine}

Die Beschleunigung der Massen ist bei der Atwoodschen Fallmaschine 
wie folgt reduziert:

\begin{equation}
 a = \frac{m_Z}{2m + m_Z} g 
\end{equation}

wobei $m_Z$ die Zusatzmasse, $ m $ eine der beiden Massen und 
$g$ die Erdbeschleunigung darstellt. 
Vorteil zur Messung der Erdbeschleunigung $g$ sind die höheren
Fallzeiten und die damit geringere Luftreibung.
So lautet der Weg $s$ abhänigig von der Beschleunigung $a$
\begin{equation}
 s = \frac{a}{2} t^2 
\end{equation}

\section{Versuchsbeschreibung}


\begin{itemize}
\item für beide Versuchsteile steht am Versuchsplatz eine 
intuitiv zu bedienende elektronische Stoppuhr
\item die Stoppuhr hat zwei Betriebsarten: freier Fall ("FF") 
und Atwoodsche Fallmaschine ("FM")
\item Rücksetzen nur möglich: Fallkörper Startkontakt geschlossen 
hat und bei der Fallmaschine auch der Stoppkontakt geschlossen ist.
\item zur Auswertung Verwendung vom Praktikumsprogramm PhysPract
\end{itemize}


\subsection{Freier Fall}


\begin{itemize}
\item Freifallapperat: Kugel fällt von einer einstellbaren Fallhöhe
\item Freifallapparat: Fallhöhe einstellbar zwischen $5cm$  und $90cm$
\item Fallhöhe abgelesen an der oberen Fläche des Kugelhalters(Pfeilmarkierung)
\end{itemize}


\subsection{Atwoodsche Fallmaschine}

\begin{itemize}
\item zwei gleiche Massen sind über eine Umlenkrolle und Faden mit einander
verbunden
\item eine Zusatzmasse an an eine der beiden Massen bewirkt eine Beschleunigung
\item die Beschleunigung ist deutlich gegeben über der Erdbeschleunigung reduziert
\item Messstrecke fest auf $s = 2m $ eingestellt
\item $m_Z,1$ bis $m_Z,11$ sind als Zusatzmassen vorhanden
\item aufgrund zufallsvertzeilter äußerer Störfaktoren ist Messung 
mindestens 5 mal zu wiederholen.
\item Bremse lösen bevor die Masse an ihre Ausgangsposition bewegt wird
\end{itemize}


\section{Versuchsdurchführung}


\subsection{Bestimmung der Erdbeschleunigung $g$ bei Vernachlässigung der Luftreibung:}


\begin{enumerate}
\item Stoppuhr auf Betreibsart freier Fall ("FF") stellen
\item Reproduzierbarkeit der Zeitmessung bei gleichbleibender Fallhöhe prüfen
\item Fallzeit $t$ für $ s = 5, 10, 15, 20, 25, 30, 35, 40, 45, 50, 55, 60, 
65, 70, 75, 80, 85, 90 cm $ einmal messen
\item aus theoretischen Grundlagen bei Vernachlässigung der Luftreibung:
\begin{equation}
 \frac{2s}{t}^2 = g 
\end{equation}
\item $2s$ und $t^2$ für die Messwerte in Excel erstellen
\item die Tabelle abspeichern
\item plotten von x-Achse $t^2$ und y-Achse $2s$ als Ausgleichsgerade
\item Anstieg entspricht der Erdbeschleunigung $g$
\item g notieren und Plot abspeichern
\end{enumerate}


\subsection{Bestimmung von Erdbeschleunigung $g$ mit Einfluss der Luftreibung:}


\begin{enumerate}
\item Berechnung von g für für $s > 40cm$ also $s = 45, 50, 55, 60, 65, 70, 75, 80, 85, 
90cm $
\item aus theoretischen Grundlagen:
\begin{equation}
c = \frac{3 rho_L}{20 rho_K r}
\end{equation}
\item für $ rho_L $ wird die aktuelle Temperatur gemessen und in Kelvin notiert.
Desweiteren wird der aktuelle Luftdruck gemessen und in Pascal notiert.
\item mit der Formel $ rho = 3.49 * 10^-3 \frac{Ks^2}{m^2} * 
\frac{p}{T} $ wird der aktuelle Luftdruck berechnet
und notiert in der Einheit $\frac{kg}{m^3}$
\end{enumerate}

- Dichte der Stahlkugel ist hier $rho_K = 7.796 * 10^3 \frac{kg}{m^3}$
- Der Radius der Stahlkugel ist hier $ r_k = 8 * 10^-3 m $
- Berechne und notiere c 
- öffne Excel und nutze die bereits erstelle Tabelle
- Brechne mit $ g = (3 + wurzel(9 - 12cs))/(ct^2) $ die Erdbeschleunigungen $g$ für alle $s > 40cm$.
- Berechne die g mit der Excel-Tabellenkalkulation und speichere die Tabelle.
- Vergleichen der Erdbeschleunigung $g$ unter Vernachlässigung und Berücksichtigung der Luftreibung.
- $g_ohne$ sei der Wert ohne Luftwiderstand und $g_mit$ sei der Wert mit Einfluss des Luftwiderstandes.
- absoluter Unterschied: $ delta g = g_ohne - g_mit $
- relativer Unterschied in %: $ \frac{g_ohne - g_mit}{g_ohne} * 100 $
- absolute und relative Unterschiede notieren


\begin{enumerate}
\item Stoppuhr auf Betreibsart Atwoodsche Fallmaschine ("FM") stellen
\item für jede der 11 Zusatzmassen die Messung 5 mal wiederholen
\item Wichtig: Bremse vor dem Rücksetzen auf die Ausgangsposition lösen
\item Messwerte mit Excel mitteln
\item aus Fallhöhe $s = 2m$ und den gemittelten $avg(t)$ mithilfe von $ a = \frac{2s}{t}^2 $
$a_1$ bis $a_11$ bestimmen
\item nutze dazu wieder Excel
\item Bestimmung der Zusatzmassen $m_Z$
\item notiere Wägesatz (A oder B)
\item Masse der beiden Fallkörper mit Faden: $2m = (370.00 plusminus 0.02)g$ 
\end{enumerate}

Ermittlung der effektiven Rollmasse $m_R,eff$:

\begin{enumerate}
\item Versuch wird nach Abb.3 aufgebaut
\item eine baugleiche Umlenkrolle, Spiralfeder und verschiedene Anhängemassen $m_P$
stehen zur Verfügung
\item Messung von Schwinugungsdauer $T$ über 10 bis 50 Schwingungsperioden (wähle eine)
\item notiere Schwingungsperioden
\item Anhängemassen varieren $m_P$ für $30, 50, 70, 90, 110, 130, 150g$
\item bestimme mit Excel $T^2$
\item eine Ausgleichsgerade mit den Programm Praktikumsprogramm PhysPract zeichnen
\item Y-Achse $T^2$ und X-Achse $m_p$
\item sinnvolle Einheiten für die Achsenbeschriftung wählen
\item Dahinter stehende Formel:
\begin{equation}
 T^2 = (4pi^2)/D(m_P + m_R,eff + \frac{m_F}{3}) 
\end{equation}
\item wobei $m_P$ die jeweiligen Anhängemassen, $m_R,eff$ die effektive Rollreibung und $m_F$ die Masse der Feder ist.
\item der Term $(4pi^2)/D (m_R,eff + \frac{m_F}{3})$ beschreibt den Schnittpunkt mit der Y-Achse also:
\begin{equation}
 y(x=0) = (4pi^2)/D (m_R,eff + \frac{m_F}{3})
\end{equation}
umgestellt nach $m_R,eff$:
\begin{equation}
 m_R,eff = (D y(x=0))/(4pi^2) - \frac{m_F}{3} 
\end{equation}
\item $y(x=0)$ aus der Ausgleichsgerade bestimmen, $m_F$ abwiegen und $D$ die Federkonstante bestimmen.
\item mit der letzen Formel die effektive Rollmasse $m_R,eff$ bestimmen
\item Plott abspeichern 
\end{enumerate}

Erdbeschleunigung $g$ und Reibungskraft $F_R$ bestimmen

\begin{enumerate}
\item $m_M = 2m + m_R,eff$ bestimmen
\item mit Excel aus den Zusatzmassen $m_Z$ und gemessenen Beschleunigungen:
\item wegen $m_Zg - F_R = (m_M + m_Z)a$ $(m_M + m_Z)a$ über die $m_z$ auftragen
\item Ausgleichsgerade plotten
\item die gesuchten Größen mit Anstieg und Absolutglied
\end{enumerate}


\section{Messwerte}

\section{Auswertung und Ergebnis}

Die mit den beiden unterschiedlichen Methoden gefundenen Werte für die Erdbeschleunigung sind,
unter Berücksichtigung der berechneten Standardunsicherheiten, mit Tabellenwerten zu vergleichen,
mögliche Abweichungen sind zu diskutieren.



\section{Fehlerdiskussion}



\begin{itemize}
\item Punkt 1
\item Punkt 2
\item Punkt 3
\end{itemize}

Display:
\begin{equation}
\kappa
\end{equation}
\begin{equation}
\frac{a}{b}
\end{equation}

klasdfjlsd
lskdfjlöska
dsölkfjasdfasöldkfjas
ökljsdfgk

\section{eine Idee}
\subsection{eine kleine Idee}
aslkdfa
sdlöfkjgsdlökf
a/b

ölkdsaflaskdjf
södlkfjasöldkfjsöalkdjfsalökdfjsaölkdfjslökdjfsaölkdjfslakdjf
äölkasdjjölkasdjfölksajdfölksadjflkösajdf
sadölkjffhasldkjfhsakdjfhsalökdfjjslkjdföalsk

$\kappa$ und $\frac{a}{b}$ und $\frac{a + b}{c + d}$

\begin{equation}
\kappa
\end{equation}
$\frac{\frac{x+1}{x-1}}{2}$


\begin{itemize}
\item Milch
\item Zucker
\end{itemize}
\end{document}
