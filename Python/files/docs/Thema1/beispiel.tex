\makeatletter
\def\input@path{{F:/Git/fantasyntax/Python/latex_klassen/}}
\makeatother

\documentclass{protokoll}

\praktikumstitel{Grundpraktikum Physik}
\versuchstitel{M2 Schwerebeschleunigung}
\autoren{Hannes Honert, Alexander Kühn}
\studiengang{TKS, EIT (DIPLOM)}
\datum{27. Mai 2025}

\begin{document}
\section{Einleitung}

\subsection{Theorie}

lsakdjhflksajdflsakjdfsalöfksaölfj
blablabla

sadnasdg \textbf{blablabla} ölasdkfjöasldkjf
noch was 

öhdsjglöksdfjölskdjfg
blablabla

\begin{equation}
\frac{a}{b}
\end{equation}


\begin{equation}
\frac{a^{e + 2} + b\cdot2}{ asldkfj }
\end{equation}


\begin{equation}
Hallo 2^{3 + 4} hallo C^\circ \approx 10 U_{Spannung} I^{Widerstandd} \sqrt{1 - U_{Spannung}}
\end{equation}


\begin{equation}
4 \plusminus 5
\end{equation}



aösdlkfjsaöldkf $ \frac{a}{b} $
asödlfkjsldkjf
a/b 

\begin{itemize}

\item Apfel
\item Banane
\item Kirsche

\end{itemize}


Zwischentext

\begin{enumerate}

\item Eins
\item Zwei
\item Drei

\end{enumerate}


\begin{spalten}

Das ist die erste Spalte

\anderespalte
Das ist die zweite Spalte

\end{spalten}


\zeilentabelle{messdaten.csv}
\grafik[0.8]{bild2}
 

Ende
\end{document}

