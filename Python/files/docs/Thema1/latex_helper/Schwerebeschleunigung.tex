\makeatletter
\def\input@path{{F:/Git/fantasyntax/Python/latex_klassen/}}
\makeatother

\documentclass{protokoll}

\praktikumstitel{Grundpraktikum Physik}
\versuchstitel{M2 Schwerebeschleunigung}
\autoren{Hannes Honert, Alexander Kühn}
\studiengang{TKS, EIT (DIPLOM)}
\datum{27. Mai 2025}

\begin{document}
\section{Aufgabenstellung}


\begin{enumerate}

\item Die Schwerebeschleunigung der Erde ist aus dem freien Fall 
einer Stahlkugel zu bestimmen. Der Einfluss der Luftreibung 
ist abzuschätzen
\item Die Schwerebeschleunigung ist weiterhin mithilfe einer Atwoodschen 
Fallmaschine zu bestimmen. Effektive Masse und Lagerreibung der 
Umlenkrolle sind anzugeben.

\end{enumerate}


\section{Theoretische Grundlagen}

Die Erdbeschleunigung $g$ liegt nach Tafelwerk bei $g = 9.80665 \frac{m}{s^2}$.




\subsection{Erdbeschleunigung ohne Luftreibung:}

Die Erdbeschleunigung $g$ wird bei Vernachlässigung der Luftreibung 

wie folgt definiert:


\begin{equation}
 g = \frac{2s}{t^2} 
\end{equation}




wobei $s$ der Weg und $t$ die Fallzeit ist.




\subsection{Erdbeschleunigung mit Luftreibung:}

Der Weg des freien Falls mit Einfluss der Luftreibung 

wird gemäß der Versuchsanleitung so definiert:


\begin{equation}
 s(t) = \frac{1}{2} g t^2 - \frac{1}{12} c g^2 t^4 
\end{equation}




umgestellt nach $2s(t)$:


\begin{equation}
 2s = g t^2 (1 - \frac{1}{6} c g t^2) 
\end{equation}




nach $g$ durch Nullstellen berechnen:


\begin{equation}
 0 = \frac{1}{2} g t^2 - \frac{1}{12} c g^2 t^4 - s 
\end{equation}



\begin{equation}
 0 = g^2 - \frac{6}{ct^2} g + \frac{12s}{ct^4} 
\end{equation}



\begin{equation}
 g_{1,2} = \frac{3}{ct^2} \plusminus \frac{\sqrt{9 - 12cs}}{ct^2} 
\end{equation}




für minus:


\begin{equation}
 g = \frac{3 - \sqrt{9 - 12cs}}{ct^2} 
\end{equation}




Eine Formel für die Erdbeschleunigung $g$ mit dem Einfluss der Luftreibung 

entsteht. Dabei wird die Konstante $c$ wird bestimmt aus:


\begin{equation}
c = \frac{3 \rho_L}{20 \rho_K r}
\end{equation}




wobei $\rho_L$ die Luftdichte, $\rho_K$ die Dichte der Kugel 
und $r$ der Radius der Kugel ist.

Die Luftdichte $\rho_L$ wird wie folgt bestimmt:

Gegeben ist die Dichte der Luft unter Normbedingungen: 
$ \rho_L (p_0, T_0) = 1.2928 \frac{kg}{m^3}, p_0 = 101.3 kPa, 
T_0 = 0 C^\circ = 273.15 K$

Die thermische Zustandsgleichung idealer Gase lautet wie folgt:

\begin{equation}
nRT = pV
\end{equation}



mit $n$ der Stoffmenge, $R$ der universellen Gaskonstante, 
$T$ der Temperatur, $p$ den Druck und $V$ den Volumen.

Die Stoffmenge ist definiert als:

\begin{equation}
 n = \frac{m}{M} 
\end{equation}



mit $m$ der Masse und $M$ der molaren Masse.

$M$ und $R$ sind konstant. $M$ ist konstant, da es sich 
um die gleiche Luft handelt. Setzt man die Gleichungen 
in einander ein und stellt nach den Konstanten um erhält man:

\begin{equation}
 \frac{R}{M} = \frac{pV}{mT} 
\end{equation}



wobei $\frac{m}{V} = \rho$ die Dichte darstellt:


\begin{equation}
 \frac{M}{R} = \frac{\rho T}{p} = konst.
\end{equation}




Setzt man nun für $ \rho = \rho_L = 1.2928 \frac{kg}{m^3}, 
T = T_0 = 0 C^\circ = 273.15 K $ und für $ p = p_0 = 101.3 kPa $
erhält man:

\begin{equation}
\frac{M}{R} = \frac{\rho_L\cdot T_0}{p_0} = \frac{1.2928 \frac{kg}{m^3}\cdot 273.15 K}{101.3 kPa}
= 3.49\cdot 10^-3 \frac{Ks^2}{m^2} 
\end{equation}



Um den aktuellen Luftdruck zu messen, wird nach $\rho$ umgestellt:

\begin{equation}
 \rho = \frac{M}{R} \frac{p}{T} 
\end{equation}



hier:

\begin{equation}
 \rho_{Luft, aktuell} = 3.49\cdot 10^{-3} \frac{Ks^2}{m^2}\cdot 
\frac{p_{Luft, aktuell}}{T_{Luft, aktuell}} 
\end{equation}



Diese Formel kommt im Versuch zum freien Fall zum Einsatz




\subsection{Atwoodsche Fallmaschine}
Die Beschleunigung der Massen ist bei der Atwoodschen Fallmaschine 
wie folgt reduziert:

\begin{equation}
 a = \frac{m_Z}{2m + m_Z} g 
\end{equation}



wobei $m_Z$ die Zusatzmasse, $ m $ eine der beiden Massen und 
$g$ die Erdbeschleunigung darstellt. 
Vorteil zur Messung der Erdbeschleunigung $g$ sind die höheren
Fallzeiten und die damit geringere Luftreibung.
So lautet der Weg $s$ abhänigig von der Beschleunigung $a$
\begin{equation}
 s = \frac{a}{2} t^2 
\end{equation}






\section{Versuchsbeschreibung}
\begin{itemize}

\item für beide Versuchsteile steht am Versuchsplatz eine 
intuitiv zu bedienende elektronische Stoppuhr
\item die Stoppuhr hat zwei Betriebsarten: freier Fall ("FF") 
und Atwoodsche Fallmaschine ("FM")
\item Rücksetzen nur möglich: Fallkörper Startkontakt geschlossen 
hat und bei der Fallmaschine auch der Stoppkontakt geschlossen ist.
\item zur Auswertung Verwendung vom Praktikumsprogramm PhysPract

\end{itemize}


\subsection{Freier Fall}
\begin{itemize}

\item Freifallapperat: Kugel fällt von einer einstellbaren Fallhöhe
\item Freifallapparat: Fallhöhe einstellbar zwischen $5cm$  und $90cm$
\item Fallhöhe abgelesen an der oberen Fläche des Kugelhalters(Pfeilmarkierung)

\end{itemize}


\subsection{Atwoodsche Fallmaschine}
\begin{itemize}

\item zwei gleiche Massen sind über eine Umlenkrolle und Faden mit einander
verbunden
\item eine Zusatzmasse an an eine der beiden Massen bewirkt eine Beschleunigung
\item die Beschleunigung ist deutlich gegeben über der Erdbeschleunigung reduziert
\item Messstrecke fest auf $s = 2m $ eingestellt
\item $m_{Z,1}$ bis $m_{Z,11}$ sind als Zusatzmassen vorhanden
\item aufgrund zufallsvertzeilter äußerer Störfaktoren ist Messung 
mindestens 5 mal zu wiederholen.
\item Bremse lösen bevor die Masse an ihre Ausgangsposition bewegt wird

\end{itemize}


\section{Versuchsdurchführung}




\subsection{Bestimmung der Erdbeschleunigung g bei Vernachlässigung der Luftreibung:}
\begin{enumerate}

\item Stoppuhr auf Betreibsart freier Fall ("FF") stellen
\item Reproduzierbarkeit der Zeitmessung bei gleichbleibender Fallhöhe prüfen
\item Fallzeit $t$ für $ s = 5, 10, 15, 20, 25, 30, 35, 40, 45, 50, 55, 60, 
65, 70, 75, 80, 85, 90 cm $ einmal messen
\item aus theoretischen Grundlagen bei Vernachlässigung der Luftreibung:
\begin{equation}
 \frac{2s}{t^2} = g 
\end{equation}


\item $2s$ und $t^2$ für die Messwerte in Excel erstellen
\item die Tabelle abspeichern
\item plotten von x-Achse $t^2$ und y-Achse $2s$ als Ausgleichsgerade
\item Anstieg entspricht der Erdbeschleunigung $g$
\item g notieren und Plot abspeichern

\end{enumerate}


Zur Reproduzierbarkeit der Zeitmessung bei gleichbleibener Fallhöhe wurden zwei Messungen M1 und M2 durchgeführt.
Es wurde die Fallzeit $t$ der Kugel aus einer Höhe von $s = 90cm$ ermittelt. \\
Für M1: $t = 0.4338s$ \\
Für M2: $t = 0.4347s$ \\
Die Werte weichen nur gering von eineinander ab mit 
$\frac{|0.4338s - 0.4347s|}{0.4338s}\cdot 100 \% = 0.21 \%$ relativen Fehler.
Anschließend wurde die Fallzeit der Kugel für $s = 5cm$ bis $s = 90cm$ in $5cm$-Schritten gemessen.
Darauf folgend wurden aus den Messwerten jeweils $2s$ und $t^2$ bestimmt und mit dem
Praktikumsprogramm PhysPract eine Ausgleichsgerade bestimmt.
Nach der Formel:


\begin{equation}
2s = g\cdot t^2
\end{equation}




kann $g$ durch den Anstieg der Ausgleichsgerade bestimmt werden.


\grafikpdf[0.8]{Messreihe zum Freifallversuch zur Bestimmung der Schwerebeschleunigung g durch Ausgleichsgerade (eigene Darstellung mit PhysPract)}


Das Diagramm zeigt die doppelte Fallhöhe $2s$ in $[cm]$ über das Quadrat der Fallzeit $t^2$ in $[s^2]$.
Der Anstieg $a$ in $[\frac{cm}{s^2}]$ der Ausgleichsgeraden stellt die Schwerebeschleunigung $g$
So lautet die Schwerebeschleunigung $g$ für den Freifallversuch mit Vernachlässigung der Luftreibung:


\begin{equation}
 g = 9.9896 \plusminus 0.156 \frac{m}{s^2} 
\end{equation}

 


Der Wert aus dem Tafelwerk lautet: $g = 9.80665 \frac{m}{s^2}$
Der absolute Fehler lautet: $ |9.9896 \frac{m}{s^2} - 9.80665 \frac{m}{s^2}|  = 0.183 \frac{m}{s^2}$
Der relative Fehler lautet: $\frac{|9.9896 \frac{m}{s^2} - 9.80665 \frac{m}{s^2}|}{9.80665 \frac{m}{s^2}}\cdot 100 \% = 1.87 \%$
der Absolute Fehler $0.183 \frac{m}{s^2}$ ist größer als die Standardabweichung $ 0.156 \frac{m}{s^2} $ 
und relative Fehler liegt bei $1.87 \%$ Im Diagramm erkennt man, dass die ersten Werte am weitesten von den Ausgleichsgerade entfernt sind. 
Es kann geschlussfolgert werden, dass man erst ab höheren Fallstrecken misst, die Messungen wiederholt oder noch höhere Fallstrecken misst,
um ein besseres Ergebnis zu bekommen. Jedoch wird der Einfluss der Luftreibung bei höheren Fallzeiten größer und könnte so das Ergebnis mehr verfälschen.


%zeilentabelle: Freifallversuch.csv


\subsection{Bestimmung von Erdbeschleunigung g mit Einfluss der Luftreibung:}
\begin{enumerate}

\item Berechnung von g für für $s > 40cm$ also $s = 45, 50, 55, 60, 65, 70, 75, 80, 85, 
90cm $
\item aus theoretischen Grundlagen:
\begin{equation}
c = \frac{3 \rho_L}{20 \rho_K r}
\end{equation}


\item für $ \rho_L $ wird die aktuelle Temperatur gemessen und in Kelvin notiert.
Desweiteren wird der aktuelle Luftdruck gemessen und in Pascal notiert.
\item mit der Formel $ \rho = 3.49\cdot 10^{-3} \frac{Ks^2}{m^2}\cdot 
\frac{p}{T} $ wird der aktuelle Luftdruck berechnet
und notiert in der Einheit $\frac{kg}{m^3}$
\item Dichte der Stahlkugel ist hier $\rho_K = 7.796\cdot 10^3 \frac{kg}{m^3}$
\item Der Radius der Stahlkugel ist hier $ r_k = 8\cdot 10^{-3} m $
\item Berechne und notiere c 
\item öffne Excel und nutze die bereits erstelle Tabelle
\item Brechne mit $ g = \frac{3 + \sqrt{9 - 12cs}}{ct^2} $ die Erdbeschleunigungen 
$g$ für alle $s > 40cm$.
\item Berechne die g mit der Excel-Tabellenkalkulation und speichere die Tabelle.
\item Vergleichen der Erdbeschleunigung $g$ unter Vernachlässigung und Berücksichtigung 
der Luftreibung.
\item $g_{ohne}$ sei der Wert ohne Luftwiderstand und $g_{mit}$ sei der Wert mit Einfluss 
des Luftwiderstandes.
\item absoluter Unterschied: $ \Delta g = g_{ohne} - g_{mit} $
\item relativer Unterschied in %: $ \frac{g_{ohne} - g_{mit}}{g_{ohne}}\cdot 100 $
\item absolute und relative Unterschiede notieren

\end{enumerate}


Um die Schwerebeschleunigung $g$ mit Einfluss der Luftreibung bei dem Freifallversuch zu bestimmen,
wird die Formel aus der Versuchsanleitung herangezogen:


\begin{equation}
g = \frac{3 - \sqrt{9 - 12cs}}{ct^2}
\end{equation}




mit der Konstanten $c$:


\begin{equation}
c = \frac{3 \rho_L}{20 \rho_K r}
\end{equation}




Da $\rho_K$ die Dichte der Kugel und $r$ der Radius der Kugel bekannt sind,
muss $\rho_L$ die Dichte der Luft noch bestimmt werden. Aus den theoretischen Grundlagen
können wir $\rho_L$ bestimmen:


\begin{equation}
\rho_L = 3.49\cdot 10^{-3} \frac{Ks^2}{m^2}\cdot \frac{p}{T}
\end{equation}




Mit den aktuellen Druck $p_a = 722 Torr \plusminus 0.5 Torr$ und der aktuellen Temperatur
$T_a = 21 \plusminus 0.5 ^\circC$ (eigene Messwerte) umgerechnet in Pascal und Kelvin kann $\rho_L$
einfach bestimmt werden: $p_a = ((722 \plusminus 0.5)\cdot \frac{101325}{760}) Pa = (96258.75 \plusminus 66.66) Pa$ 
und $T_a = (21 \plusminus 0.5 + 273.15) K = (294.15 \plusminus 0.5) K$


\begin{equation}
 \rho_L = 3.49\cdot 10^{-3} \frac{Ks^2}{m^2}\cdot \frac{96258.75 Pa}{294.15 K} = 1.1421 \frac{kg}{m^2}
\end{equation}




mit $\rho_K = 7.796\cdot 10^3 \frac{kg}{m^3}$ und $ r_k = 8\cdot 10^{-3} m$ (aus Versuchsanleitung)
kann die Konstante $c$ bestimmt werden:


\begin{equation}
c = \frac{3 \rho_L}{20 \rho_K r} = \frac{3\cdot 1.1421 \frac{kg}{m^2}}{20\cdot 7.796\cdot 10^3 \frac{kg}{m^3}\cdot 8\cdot 10^{-3} m}
= 2.747\cdot 10^{-3} \frac{1}{m}
\end{equation}




So lautet $g$:


\begin{equation}
g = \frac{3 - \sqrt{9 - 12cs}}{ct^2} = \frac{3 - \sqrt{9 - 12\cdot 2.747\cdot 10^{-3} \frac{1}{m}\cdot s}}{2.747\cdot 10^{-3} \frac{1}{m}\cdot t^2}
\end{equation}




%zeilentabelle: Freifallversuch_mit_Luftreibung.csv


Mit $ \frac{2s}{t^2} = g $ können aus den Messwerten aus dem Freifallversuch ohne Einfluss der 
Luftreibung die einzelnen Schwerebeschleungigungen errechnet werden.
Diese werden dann mit den Schwerebeschleunigungen aus Betrachtung mit Einfluss der Luftreibung verglichen.

%zeilentabelle: Freifallversuch_Fehlerbetrachtung.csv

Man erkennt, dass der relative Fehler gering mit maximal ca. $0.08 \%$ ist.
Er steigt, jedoch von der Fallstrecke $s = 40cm$ bis $s = 90cm$ streng monoton 
bis auf mehr als das Doppelte, von $0.0367 \%$ auf $0.0825 \%$.


\begin{enumerate}

\item Stoppuhr auf Betreibsart Atwoodsche Fallmaschine ("FM") stellen
\item für jede der 11 Zusatzmassen die Messung 5 mal wiederholen
\item Wichtig: Bremse vor dem Rücksetzen auf die Ausgangsposition lösen
\item Messwerte mit Excel mitteln
\item aus Fallhöhe $s = 2m$ und den gemittelten $\bar{t}$ mithilfe von $ a = \frac{2s}{t^2} $
$a_1$ bis $a_{11}$ bestimmen
\item nutze dazu wieder Excel
\item Bestimmung der Zusatzmassen $m_Z$
\item notiere Wägesatz (A oder B)
\item Masse der beiden Fallkörper mit Faden: $2m = (370.00 \plusminus 0.02)g$ 

\end{enumerate}


Ermittlung der effektiven Rollmasse $m_{R,eff}$:

\begin{enumerate}

\item Versuch wird nach Abb.3 aufgebaut
\item eine baugleiche Umlenkrolle, Spiralfeder und verschiedene Anhängemassen $m_P$
stehen zur Verfügung
\item Messung von Schwinugungsdauer $T$ über 10 bis 50 Schwingungsperioden (wähle eine)
\item notiere Schwingungsperioden
\item Anhängemassen varieren $m_P$ für $30, 50, 70, 90, 110, 130, 150g$
\item bestimme mit Excel $T^2$
\item eine Ausgleichsgerade mit den Programm Praktikumsprogramm PhysPract zeichnen
\item Y-Achse $T^2$ und X-Achse $m_p$
\item sinnvolle Einheiten für die Achsenbeschriftung wählen
\item Dahinter stehende Formel:

\begin{equation}
 T^2 = \frac{4\pi^2}{D(m_P + m_{R,eff} + \frac{m_F}{3})} 
\end{equation}


\item wobei $m_P$ die jeweiligen Anhängemassen, $m_{R,eff}$ die effektive Rollreibung 
und $m_F$ die Masse der Feder ist.
\item der Term $\frac{4\pi^2}{D(m_{R,eff} + \frac{m_F}{3})}$ beschreibt den Schnittpunkt mit der Y-Achse also:
\begin{equation}
 y(x=0) = \frac{4\pi^2}{D(m_R,eff + \frac{m_F}{3})}
\end{equation}


umgestellt nach $m_R,eff$:
\begin{equation}
 m_R,eff = \frac{D y(x=0)}{4\pi^2} - \frac{m_F}{3} 
\end{equation}


\item $y(x=0)$ aus der Ausgleichsgerade bestimmen, $m_F$ abwiegen und $D$ die Federkonstante bestimmen.
\item mit der letzen Formel die effektive Rollmasse $m_R,eff$ bestimmen
\item Plott abspeichern 

\end{enumerate}


Erdbeschleunigung $g$ und Reibungskraft $F_R$ bestimmen

\begin{enumerate}

\item $m_M = 2m + m_{R,eff}$ bestimmen
\item mit Excel aus den Zusatzmassen $m_Z$ und gemessenen Beschleunigungen:
\item wegen $m_Zg - F_R = (m_M + m_Z)a$ $(m_M + m_Z)a$ über die $m_z$ auftragen
\item Ausgleichsgerade plotten
\item die gesuchten Größen mit Anstieg und Absolutglied

\end{enumerate}


\section{Messwerte}



\section{Auswertung und Ergebnis}
Die mit den beiden unterschiedlichen Methoden gefundenen Werte für die Erdbeschleunigung sind,
unter Berücksichtigung der berechneten Standardunsicherheiten, mit Tabellenwerten zu vergleichen,
mögliche Abweichungen sind zu diskutieren.




\section{Fehlerdiskussion}



\begin{itemize}

\item Punkt 1
\item Punkt 2
\item Punkt 3

Display:
\begin{equation}
kappa
\end{equation}


\begin{equation}
\frac{a}{b}
\end{equation}



klasdfjlsd
lskdfjlöska
dsölkfjasdfasöldkfjas
ökljsdfgk

\section{eine Idee }
\subsection{eine kleine Idee}
aslkdfa
sdlöfkjgsdlökf
a/b

ölkdsaflaskdjf
södlkfjasöldkfjsöalkdjfsalökdfjsaölkdfjslökdjfsaölkdjfslakdjf
äölkasdjjölkasdjfölksajdfölksadjflkösajdf
sadölkjffhasldkjfhsakdjfhsalökdfjjslkjdföalsk

$kappa$ und $\frac{a}{b}$ und $\frac{a + b}{c + d}$

\begin{equation}
kappa
\end{equation}


$\frac{\frac{x+1}{x-1}}{2}$

\end{itemize}


liste:
- Milch
- Zucker

\end{document}

