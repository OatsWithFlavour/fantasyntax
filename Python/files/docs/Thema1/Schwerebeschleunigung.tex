\makeatletter
\def\input@path{{F:/Git/fantasyntax/Python/latex_klassen/}}
\makeatother

\documentclass{protokoll}

\praktikumstitel{Grundpraktikum Physik}
\versuchstitel{M2 Schwerebeschleunigung}
\autoren{Hannes Honert, Alexander Kühn}
\studiengang{TKS, EIT (DIPLOM)}
\datum{27. Mai 2025}

\begin{document}
\section{Aufgabenstellung}


\begin{enumerate}

\item Die Schwerebeschleunigung der Erde ist aus dem freien Fall 
einer Stahlkugel zu bestimmen. Der Einfluss der Luftreibung 
ist abzuschätzen
\item Die Schwerebeschleunigung ist weiterhin mithilfe einer Atwoodschen 
Fallmaschine zu bestimmen. Effektive Masse und Lagerreibung der 
Umlenkrolle sind anzugeben.

\end{enumerate}


\section{Theoretische Grundlagen}

Die Erdbeschleunigung $g$ liegt nach Tafelwerk bei $g = 9.80665 \frac{m}{s^2}$.




\subsection{Erdbeschleunigung ohne Luftreibung:}

Die Erdbeschleunigung $g$ wird bei Vernachlässigung der Luftreibung 

wie folgt definiert:


\begin{equation}
 g = \frac{2s}{t^2} 
\end{equation}




wobei $s$ der Weg und $t$ die Fallzeit ist.




\subsection{Erdbeschleunigung mit Luftreibung:}

Der Weg des freien Falls mit Einfluss der Luftreibung 

wird gemäß der Versuchsanleitung so definiert:


\begin{equation}
 s(t) = \frac{1}{2} g t^2 - \frac{1}{12} c g^2 t^4 
\end{equation}




umgestellt nach $2s(t)$:


\begin{equation}
 2s = g t^2 (1 - \frac{1}{6} c g t^2) 
\end{equation}




nach $g$ durch Nullstellen berechnen:


\begin{equation}
 0 = \frac{1}{2} g t^2 - \frac{1}{12} c g^2 t^4 - s 
\end{equation}



\begin{equation}
 0 = g^2 - \frac{6}{ct^2} g + \frac{12s}{ct^4} 
\end{equation}



\begin{equation}
 g_{1,2} = \frac{3}{ct^2} \plusminus \frac{\sqrt{9 - 12cs}}{ct^2} 
\end{equation}




für minus:


\begin{equation}
 g = \frac{3 - \sqrt{9 - 12cs}}{ct^2} 
\end{equation}




Eine Formel für die Erdbeschleunigung $g$ mit dem Einfluss der Luftreibung 

entsteht. Dabei wird die Konstante $c$ wird bestimmt aus:


\begin{equation}
c = \frac{3 \rho_L}{20 \rho_K r}
\end{equation}




wobei $\rho_L$ die Luftdichte, $\rho_K$ die Dichte der Kugel 
und $r$ der Radius der Kugel ist.

Die Luftdichte $\rho_L$ wird wie folgt bestimmt:

Gegeben ist die Dichte der Luft unter Normbedingungen: 
$ \rho_L (p_0, T_0) = 1.2928 \frac{kg}{m^3}, p_0 = 101.3 kPa, 
T_0 = 0 C^\circ = 273.15 K$

Die thermische Zustandsgleichung idealer Gase lautet wie folgt:

\begin{equation}
nRT = pV
\end{equation}



mit $n$ der Stoffmenge, $R$ der universellen Gaskonstante, 
$T$ der Temperatur, $p$ den Druck und $V$ den Volumen.

Die Stoffmenge ist definiert als:

\begin{equation}
 n = \frac{m}{M} 
\end{equation}



mit $m$ der Masse und $M$ der molaren Masse.

$M$ und $R$ sind konstant. $M$ ist konstant, da es sich 
um die gleiche Luft handelt. Setzt man die Gleichungen 
in einander ein und stellt nach den Konstanten um erhält man:

\begin{equation}
 \frac{R}{M} = \frac{pV}{mT} 
\end{equation}



wobei $\frac{m}{V} = \rho$ die Dichte darstellt:


\begin{equation}
 \frac{M}{R} = \frac{\rho T}{p} = konst.
\end{equation}




Setzt man nun für $ \rho = \rho_L = 1.2928 \frac{kg}{m^3}, 
T = T_0 = 0 C^\circ = 273.15 K $ und für $ p = p_0 = 101.3 kPa $
erhält man:

\begin{equation}
\frac{M}{R} = \frac{\rho_L\cdot T_0}{p_0} = \frac{1.2928 \frac{kg}{m^3}\cdot 273.15 K}{101.3 kPa}
= 3.49\cdot 10^-3 \frac{Ks^2}{m^2} 
\end{equation}



Um den aktuellen Luftdruck zu messen, wird nach $\rho$ umgestellt:

\begin{equation}
 \rho = \frac{M}{R} \frac{p}{T} 
\end{equation}



hier:

\begin{equation}
 \rho_{Luft, aktuell} = 3.49\cdot 10^{-3} \frac{Ks^2}{m^2}\cdot 
\frac{p_{Luft, aktuell}}{T_{Luft, aktuell}} 
\end{equation}



Diese Formel kommt im Versuch zum freien Fall zum Einsatz




\subsection{Atwoodsche Fallmaschine}


Über eine Rolle sind zwei Massen mit einem Seil verbunden.
Die Massen sollen zunächst gleich sein. Es bewegt sich nichts. Es wird eine
Zusatzmasse an eine der beiden Massen gehängt. Es beginn sich zu bewegen.
Die Beschleunigung der Massen ist bei der Atwoodschen Fallmaschine 
wie folgt reduziert:


\begin{equation}
 a = \frac{m_Z}{2m + m_Z} g 
\end{equation}




wobei $m_Z$ die Zusatzmasse, $ m $ eine der beiden Massen und 
$g$ die Erdbeschleunigung darstellt. 
Vorteil zur Messung der Erdbeschleunigung $g$ sind die höheren
Fallzeiten und die damit geringere Luftreibung.
So lautet der Weg $s$ abhänigig von der Beschleunigung $a$


\begin{equation}
 s = \frac{a}{2} t^2 
\end{equation}




Die zusätzliche Masse verschiebt, das System aus dem statischen Kräftegleichgewicht.
Die Erdanziehungskraft wirkt auf die beiden Massen. Da die eine Masse größer ist als die andere,
bewegt sich das Seil in Richtung der größeren Masse. Entgegeben diese Kraft wirkt die Reibungskraft $F_R$. Daraus folgt:


\begin{equation}
 F = ma = g(m + m_Z) - F_R - gm 
\end{equation}


\begin{equation}
 F = ma = gm_Z - F_R 
\end{equation}




Die Bewegung wird durch die Massen und die Trägheit der Rolle gehempt.
Die Trägheit der Rolle kann auch als effektive Masse $m_{R,eff}$ angesehen werden, die die Bewegung hemt.
Daraus folgt:

\begin{equation}
 F = ma = (2m + m_Z + m_{R,eff})a = gm_Z - F_R 
\end{equation}




Schwingungsexperiment: Um diese effektive Masse zu bestimmen wird ein Schwingungsexperiment durchgeführt, wobei eine
Seite der Rolle über eine Feder mit dem Boden verbunden wird und die andere Seite mit einem Probegewicht $m_P$
in Schwingung gebracht. Die Quadrat der Schwingungsdauer berechnet sich aus:


\begin{equation}
 T^2 = \frac{4 \pi}{D} (m_P + m_{R,eff} + \frac{m_F}{3}) 
\end{equation}




für $T=0$ kann $m_{R,eff}$ einfach bestimmt werden:


\begin{equation}
 0 = \frac{4 \pi}{D} (m_P(T=0) + m_{R,eff} + \frac{m_F}{3}) 
\end{equation}


\begin{equation}
 0 = m_P(T=0) + m_{R,eff} + \frac{m_F}{3} 
\end{equation}


\begin{equation}
 m_{R,eff} = -(m_P(T=0) + \frac{m_F}{3}) 
\end{equation}




Wenn $(2m + m_Z + m_{R,eff})a$ für verschiedene $m_Z$ eine Ausgleichsgerade erzeugt wird,
kann nun $g$ als Anstieg und $F_R$ als absolutes Glied abgelesen werden.



\section{Versuchsbeschreibung}
\begin{itemize}

\item für beide Versuchsteile steht am Versuchsplatz eine 
intuitiv zu bedienende elektronische Stoppuhr
\item die Stoppuhr hat zwei Betriebsarten: freier Fall ("FF") 
und Atwoodsche Fallmaschine ("FM")
\item Rücksetzen nur möglich: Fallkörper Startkontakt geschlossen 
hat und bei der Fallmaschine auch der Stoppkontakt geschlossen ist.
\item zur Auswertung Verwendung vom Praktikumsprogramm PhysPract

\end{itemize}


\subsection{Freier Fall}
\begin{itemize}

\item Freifallapperat: Kugel fällt von einer einstellbaren Fallhöhe
\item Freifallapparat: Fallhöhe einstellbar zwischen $5cm$  und $90cm$
\item Fallhöhe abgelesen an der oberen Fläche des Kugelhalters(Pfeilmarkierung)

\end{itemize}


\subsection{Atwoodsche Fallmaschine}
\begin{itemize}

\item zwei gleiche Massen sind über eine Umlenkrolle und Faden mit einander
verbunden
\item eine Zusatzmasse an an eine der beiden Massen bewirkt eine Beschleunigung
\item die Beschleunigung ist deutlich gegenüber der Erdbeschleunigung reduziert
\item Messstrecke fest auf $s = 2m $ eingestellt
\item $m_{Z,1}$ bis $m_{Z,11}$ sind als Zusatzmassen vorhanden
\item aufgrund zufallsvertzeilter äußerer Störfaktoren ist Messung 
mindestens 5 mal zu wiederholen.
\item Bremse lösen bevor die Masse an ihre Ausgangsposition bewegt wird

\end{itemize}


\section{Versuchsdurchführung}




\subsection{Bestimmung der Erdbeschleunigung g bei Vernachlässigung der Luftreibung:}



Zur Reproduzierbarkeit der Zeitmessung bei gleichbleibener Fallhöhe wurden zwei Messungen M1 und M2 durchgeführt.
Es wurde die Fallzeit $t$ der Kugel aus einer Höhe von $s = 90cm$ ermittelt. \\
Für M1: $t = 0.4338s$ \\
Für M2: $t = 0.4347s$ \\
Die Werte weichen nur gering von eineinander ab mit 
$\frac{|0.4338s - 0.4347s|}{0.4338s}\cdot 100 \% = 0.21 \%$ relativen Fehler.
Anschließend wurde die Fallzeit der Kugel für $s = 5cm$ bis $s = 90cm$ in $5cm$-Schritten gemessen.
Darauf folgend wurden aus den Messwerten jeweils $2s$ und $t^2$ bestimmt und mit dem
Praktikumsprogramm PhysPract eine Ausgleichsgerade bestimmt.
Nach der Formel:


\begin{equation}
2s = g\cdot t^2
\end{equation}




kann $g$ durch den Anstieg der Ausgleichsgerade bestimmt werden.


\grafikpdf[0.8]{Messreihe zum Freifallversuch zur Bestimmung der Schwerebeschleunigung g durch Ausgleichsgerade (eigene Darstellung mit PhysPract)}


Das Diagramm zeigt die doppelte Fallhöhe $2s$ in $[cm]$ über das Quadrat der Fallzeit $t^2$ in $[s^2]$.
Der Anstieg $a$ in $[\frac{cm}{s^2}]$ der Ausgleichsgeraden stellt die Schwerebeschleunigung $g$
So lautet die Schwerebeschleunigung $g$ für den Freifallversuch mit Vernachlässigung der Luftreibung:


\begin{equation}
 g = 9.9896 \plusminus 0.156 \frac{m}{s^2} 
\end{equation}

 


Der Wert aus dem Tafelwerk lautet: $g = 9.80665 \frac{m}{s^2}$
Der absolute Fehler lautet: $ |9.9896 \frac{m}{s^2} - 9.80665 \frac{m}{s^2}|  = 0.183 \frac{m}{s^2}$
Der relative Fehler lautet: $\frac{|9.9896 \frac{m}{s^2} - 9.80665 \frac{m}{s^2}|}{9.80665 \frac{m}{s^2}}\cdot 100 \% = 1.87 \%$
der Absolute Fehler $0.183 \frac{m}{s^2}$ ist größer als die Standardabweichung $ 0.156 \frac{m}{s^2} $ 
und relative Fehler liegt bei $1.87 \%$ Im Diagramm erkennt man, dass die ersten Werte am weitesten von den Ausgleichsgerade entfernt sind. 
Es kann geschlussfolgert werden, dass man erst ab höheren Fallstrecken misst, die Messungen wiederholt oder noch höhere Fallstrecken misst,
um ein besseres Ergebnis zu bekommen. Jedoch wird der Einfluss der Luftreibung bei höheren Fallzeiten größer und könnte so das Ergebnis mehr verfälschen.


%zeilentabelle: Freifallversuch.csv


\subsection{Bestimmung von Erdbeschleunigung g mit Einfluss der Luftreibung:}


Um die Schwerebeschleunigung $g$ mit Einfluss der Luftreibung bei dem Freifallversuch zu bestimmen,
wird die Formel aus der Versuchsanleitung herangezogen:


\begin{equation}
g = \frac{3 - \sqrt{9 - 12cs}}{ct^2}
\end{equation}




mit der Konstanten $c$:


\begin{equation}
c = \frac{3 \rho_L}{20 \rho_K r}
\end{equation}




Da $\rho_K$ die Dichte der Kugel und $r$ der Radius der Kugel bekannt sind,
muss $\rho_L$ die Dichte der Luft noch bestimmt werden. Aus den theoretischen Grundlagen
können wir $\rho_L$ bestimmen:


\begin{equation}
\rho_L = 3.49\cdot 10^{-3} \frac{Ks^2}{m^2}\cdot \frac{p}{T}
\end{equation}




Mit den aktuellen Druck $p_a = 722 Torr \plusminus 0.5 Torr$ und der aktuellen Temperatur
$T_a = 21 \plusminus 0.5 ^\circ C$ (eigene Messwerte) umgerechnet in Pascal und Kelvin kann $\rho_L$
einfach bestimmt werden: $p_a = ((722 \plusminus 0.5)\cdot \frac{101325}{760}) Pa = (96258.75 \plusminus 66.66) Pa$ 
und $T_a = (21 \plusminus 0.5 + 273.15) K = (294.15 \plusminus 0.5) K$


\begin{equation}
 \rho_L = 3.49\cdot 10^{-3} \frac{Ks^2}{m^2}\cdot \frac{96258.75 Pa}{294.15 K} = 1.1421 \frac{kg}{m^2}
\end{equation}




mit $\rho_K = 7.796\cdot 10^3 \frac{kg}{m^3}$ und $ r_k = 8\cdot 10^{-3} m$ (aus Versuchsanleitung)
kann die Konstante $c$ bestimmt werden:


\begin{equation}
c = \frac{3 \rho_L}{20 \rho_K r} = \frac{3\cdot 1.1421 \frac{kg}{m^2}}{20\cdot 7.796\cdot 10^3 \frac{kg}{m^3}\cdot 8\cdot 10^{-3} m}
= 2.747\cdot 10^{-3} \frac{1}{m}
\end{equation}




So lautet $g$:


\begin{equation}
g = \frac{3 - \sqrt{9 - 12cs}}{ct^2} = \frac{3 - \sqrt{9 - 12\cdot 2.747\cdot 10^{-3} \frac{1}{m}\cdot s}}{2.747\cdot 10^{-3} \frac{1}{m}\cdot t^2}
\end{equation}




%zeilentabelle: Freifallversuch_mit_Luftreibung.csv


Mit $ \frac{2s}{t^2} = g $ können aus den Messwerten aus dem Freifallversuch ohne Einfluss der 
Luftreibung die einzelnen Schwerebeschleungigungen errechnet werden.
Diese werden dann mit den Schwerebeschleunigungen aus Betrachtung mit Einfluss der Luftreibung verglichen.

%zeilentabelle: Freifallversuch_Fehlerbetrachtung.csv

Man erkennt, dass der relative Fehler gering mit maximal ca. $0.08 \%$ ist.
Er steigt, jedoch von der Fallstrecke $s = 40cm$ bis $s = 90cm$ streng monoton 
bis auf mehr als das Doppelte, von $0.0367 \%$ auf $0.0825 \%$.


\subsection{Bestimmung von Erdbeschleunigung g mit der Atwoodschen Fallmaschiene}



Die Messstrecke ist auf eine Fallhöhe von $s = 2m$ fest eingestellt. Aus der Versuchanleitung ist die Masse der 
beiden Fallkörper: $2m = (370.00 \plusminus 0.02)g$ entnehmbar. Für die Gewichte wurde Wägesatz A verwendet.
Für jedes der 11 Gewichte $m_Z$ von $1g$ bis $11g$ im $1g$ Schritten wurden jeweils 5 Messungen durchgeführt 
und anschließend die arithmetrischen Mittelwerte aus den 5 Messungen ermittelt. Aus den Mittelwerten wurden 
die einzelnen Beschleunigungen berechnet.

Zur Berechnung der effektiven Masse $m_{R,eff}$ wurde das Schwinungsexperiment durchgeführt.
Für 10 Schwingungsperioden wurden dann Schwingungsdauern T aufgenommen und die jeweiligen
Probemassen $m_P$ von $30, 50, 70, 90, 110, 130, 150g$ gewogen.
Durch eine Ausgleichsgerade mit den Quadrat der Schwingungsdauern $T^2$ als Wert und
der Probegewichte $m_P$ als Stellen kann die effektive Masse $m_{R,eff}$ bestimmt werden.
Dazu wird der Schnittpunkt mit der $m_P$-Achse bei $T^2 = 0$ verwendet:

\begin{equation}
 m_{R,eff} = -(m_P(T=0) + \frac{m_F}{3}) 
\end{equation}



Ablesbar aus der eigenen Darstellung mit dem Programm PhysPract kann $m_P(T=0) = (-106.4 \plusminus 7.11) g$ ermittelt werden.
Nach eigener Messung beträgt das Gewicht der Feder $m_F = 7.02g$.
So lautet $m_{R,eff}$:


\begin{equation}
 m_{R,eff} = -(m_P(T=0) + \frac{m_F}{3}) = - (-106.4 \plusminus 7.11)g - (7.02g)/3 = (104,6 \plusminus 7.11)g 
\end{equation}



Mit $ m_M = 2m + m_{R,eff} = 370.00g + 104.6 = 474.6g$ kann eine Ausgleichsgerade für $g$ und $F_R$ ermittelt werden:

\begin{equation}
 F = ma = (m_M + m_Z)a = m_Z\cdot g - F_R 
\end{equation}



Das eigene Diagramm aufgenommen mit PhysPract zeigt die Ausgleichsgerade von $F = (m_M + m_Z)a$ in $[mN]$ für 
alle $m_Z$ in $[g]$ von $1$ bis $11g$. Die Schwerebeschleunigung ist aus dem Anstieg 
$a = g = (8.9444 \plusminus 0,278) \frac{m}{s^2}$ ablesbar Die Reibungskraft $F_R$ an der Rolle 
ist beim absoluten Glied mit $b = F_r = (85.341 \plusminus 1.88) mN$ ablesbar.




\section{Messwerte}


\grafikpdf[0.6]{Tabelle für die berechnete Schwerebescheunigung g mit Einfluss der Luftreibung beim Freifallversuch}
\grafikpdf[0.6]{Tabelle für die berechnete Schwerebeschleunigung g ohne Einfluss der Luftreibung, absoluter Fehler und relativer Fehler durch Einfluss der Luftreibung beim Freifallversuch (eigene Messung)}
\grafikpdf[0.6]{Tabelle für die Fallzeit über die Fallhöhe beim Freifallversuch (eigene Messung)}

\grafikpdf[0.6]{Tabelle für die Masse der Probemassen und die Schwingungsdauer über 10 Schwingungen (eigene Messung)}
\grafikpdf[0.6]{Tabelle für die Masse der Zusatzgewichte, die durchschnittliche Fallzeit, Beschleunigung von a1 bis a11 und der Kraft F von a(mM + mZ) (eigene Messung)}


\section{Auswertung und Ergebnis}

Die Erdbeschleunigung $g$ wurde mit zwei verschiedenen Methoden bestimmt: durch den freien Fall einer Stahlkugel sowie mithilfe der Atwoodschen Fallmaschine.

Zunächst wurde $g$ bei Vernachlässigung der Luftreibung mithilfe einer linearen Ausgleichsgeraden über die Beziehung $2s = g\cdot t^2$ berechnet. Die entsprechende Darstellung zeigt den Zusammenhang zwischen doppelter Fallhöhe $2s$ und dem Quadrat der Fallzeit $t^2$:

\grafikpdf[0.8]{Messreihe zum Freifallversuch zur Bestimmung der Schwerebeschleunigung g durch Ausgleichsgerade (eigene Darstellung mit PhysPract)}

Der Anstieg der Ausgleichsgeraden entspricht der Erdbeschleunigung:

\begin{equation}
 g = 9.9896 \plusminus 0.156 \frac{m}{s^2}
\end{equation}

Verglichen mit dem Tabellenwert $g = 9.80665 \frac{m}{s^2}$ ergibt sich ein absoluter Fehler von $0.183 \frac{m}{s^2}$ sowie ein relativer Fehler von $1{,}87\%$, der oberhalb der Standardabweichung liegt. Im Diagramm ist erkennbar, dass insbesondere bei kleinen Fallhöhen deutliche Abweichungen zur Ausgleichsgeraden auftreten. Dies legt nahe, dass höhere Fallhöhen für verlässlichere Ergebnisse vorteilhaft sind – wobei jedoch der Einfluss der Luftreibung zunimmt.

Zur Berücksichtigung der Luftreibung wurde $g$ zusätzlich unter Einbezug der Reibungskonstante $c$ berechnet. Die berechneten Werte zeigten im Vergleich zur vereinfachten Methode einen nur geringen Unterschied (maximal ca. $0{,}08\%$). Der relative Fehler stieg dabei mit zunehmender Fallhöhe streng monoton.

Anschließend wurde $g$ mit der Atwoodschen Fallmaschine bestimmt. Dazu wurden Mittelwerte aus jeweils fünf Messungen pro Zusatzmasse $m_Z$ gebildet. Die folgende Darstellung zeigt die Ausgleichsgerade für $(m_M + m_Z) \cdot a$ gegen $m_Z$, aus der $g$ und die Reibungskraft $F_R$ bestimmt wurden:

\grafikpdf[0.8]{Messreihe zum Atwoodschen Fallmaschine zur Bestimmung der Schwerebeschleunigung g und Reibungskraft FR durch Ausgleichsgerade (eigene Darstellung mit PhysPract)}

Die ermittelten Werte lauten:

\begin{itemize}
  \item Erdbeschleunigung: $g = (8.9444 \plusminus 0.278)\,\frac{m}{s^2}$
  \item Reibungskraft: $F_R = (85.341 \plusminus 1.88)\,\mathrm{mN}$
\end{itemize}

Die Abweichung zum Literaturwert lässt sich durch Reibungsverluste sowie die vergleichsweise geringe Systembeschleunigung erklären.

Für die Berechnung der effektiven Rollenmasse $m_{R,eff}$ wurde ein Schwingungsexperiment durchgeführt. Aus der Ausgleichsgeraden der Schwingungsdauern $T^2$ über den angehängten Massen $m_P$ wurde $m_P(T=0)$ bestimmt. Daraus ergibt sich:

\begin{equation}
 m_{R,eff} = -(m_P(T=0) + \frac{m_F}{3}) = (104{,}6 \plusminus 7{,}11)\,g
\end{equation}

Die zugehörige grafische Darstellung zeigt die ermittelte Ausgleichsgerade des Pendelversuchs:

\grafikpdf[0.8]{Messreihe zum Atwoodschen Fallmaschine zur Bestimmung von der effektiven Masse durch Ausgleichsgerade (eigene Darstellung mit PhysPract)}

Zusammenfassend konnten durch beide Methoden plausible Werte für die Erdbeschleunigung bestimmt werden. Die Methode über den freien Fall lieferte genauere Resultate, während die Atwoodsche Fallmaschine zusätzlich den Einfluss mechanischer Reibung quantifizierte.




\section{Fehlerdiskussion}

Bei der Auswertung des Versuchs traten sowohl zufällige als auch systematische Fehler auf, die die Genauigkeit der ermittelten Werte für die Erdbeschleunigung $g$ beeinflussen.

Im Fall des freien Falls wurde $g$ über eine Ausgleichsgerade aus den gemessenen Fallzeiten $t$ und Höhen $s$ bestimmt. Die Messgenauigkeit der Stoppuhr (Skalenwert $0{,}1\,\text{s}$) sowie mögliche Reaktionszeiten beim Starten bzw. Stoppen des Falls beeinflussen die Unsicherheit von $t$ direkt. Da $g$ gemäß $g = \frac{2s}{t^2}$ berechnet wird, pflanzen sich Fehler in $t$ nichtlinear auf $g$ fort. Die Unsicherheit von $g$ ergibt sich aus der Fehlerfortpflanzung:

\begin{equation}
\Delta g = \left| \frac{\partial g}{\partial t} \right| \cdot \Delta t = \left| -\frac{4s}{t^3} \right| \cdot \Delta t
\end{equation}

Ein kleiner Fehler in $t$ führt damit bei kurzen Fallzeiten zu einem relativ großen Fehler in $g$. Dieser Effekt ist besonders bei kleinen Fallhöhen ausgeprägt, was durch die auffällige Streuung der ersten Messpunkte in der Grafik belegt wird.

Bei der Berechnung von $g$ unter Berücksichtigung der Luftreibung wurde zusätzlich die Unsicherheit in der Luftdichte $\rho_L$ berücksichtigt. Diese wurde aus dem Luftdruck $p$ und der Temperatur $T$ berechnet, die jeweils eigene Unsicherheiten aufweisen. Aufgrund der Multiplikation und Division in der Formel

\begin{equation}
\rho_L = 3.49\cdot 10^{-3} \cdot \frac{p}{T}
\end{equation}

ergibt sich für $\Delta \rho_L$ die relative Unsicherheit aus der quadratischen Summe der relativen Unsicherheiten von $p$ und $T$. Diese fließt wiederum in die Berechnung der Konstante $c$ ein und beeinflusst damit die Korrekturformel für $g$.

Bei der Atwoodschen Fallmaschine wirkt sich zusätzlich die Unsicherheit der Massen $m$, $m_Z$ und insbesondere $m_{R,eff}$ aus. Die effektive Rollenmasse wurde mithilfe eines Schwingungsexperiments bestimmt, bei dem $m_{R,eff}$ indirekt über die lineare Ausgleichsrechnung gewonnen wurde. Die Unsicherheit von $m_{R,eff}$ ergibt sich dabei ebenfalls über Fehlerfortpflanzung:

\begin{equation}
\Delta m_{R,eff} = \sqrt{ (\Delta m_P)^2 + \left( \frac{1}{3} \Delta m_F \right)^2 }
\end{equation}

Die hohe Unsicherheit bei $m_P(T=0) = (-106{,}4 \pm 7{,}11)\,g$ schlägt sich direkt in $\Delta m_{R,eff}$ nieder und beeinflusst somit die Genauigkeit der Ausgleichsgeraden zur Bestimmung von $g$ und $F_R$.

Zusätzlich wurde bei der Berechnung von $g$ über $(m_M + m_Z) \cdot a = m_Z \cdot g - F_R$ der Einfluss der Streuung in den Beschleunigungswerten $a$ berücksichtigt. Da $a$ über $s = \frac{a}{2} t^2$ berechnet wurde, wirken sich auch hier Unsicherheiten in $t$ direkt auf $a$ und in Folge auf $g$ und $F_R$ aus.

Zusammenfassend lässt sich sagen:
\begin{itemize}
  \item Die Ergebnisse des Freifallversuchs unter Vernachlässigung der Luftreibung zeigen eine gute Übereinstimmung mit dem Literaturwert, wobei kleinere Fallhöhen größere relative Unsicherheiten verursachen.
  \item Die Berücksichtigung der Luftreibung zeigt nur geringe Unterschiede, was durch die kleine Reibungskonstante $c$ erklärbar ist.
  \item Der Versuch mit der Atwoodschen Fallmaschine zeigt eine deutlich größere Unsicherheit, was vor allem auf die indirekte Bestimmung von $m_{R,eff}$ und die Reibungseinflüsse zurückzuführen ist.
\end{itemize}




\end{document}

